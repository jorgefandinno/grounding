% ----------------------------------------------------------------------
\begin{frame}{Bottom up grounding}
  \vspace*{-10pt}\small%
  \begin{itemize}
    \item<1-> \structure{Idea} \
      Ground \alert{relevant} rules by gradually extending the atom base
    \item<2-> \structure{Relative ground instantiation} \ of a logic program $P$
      \par wrt a set of ground atoms $D$
      \begin{align*}
        \cground{P}{D} & = \{ r \in \ground{P} \mid \alert{\pbody{r} \subseteq D} \}
      \end{align*}
    \item<only@3-> \structure{Algorithm}

      \EmphAlgo{0.8}{
        \Fn{\(\GroundBottomUp(P,D)\)}{
          \(G \leftarrow \cground{P}{D}\)\;
          \If{\(\Head{G} \not\subseteq D\)}{
            \Return \(\GroundBottomUp(P,D \alert{\cup \Head{G}})\)\;
          }
          \Return \(G\)\;
        }
      }

    \item<only@4-> \structure{Property} \ Given a safe normal program \(P\) and set of ground facts \(I\),

      \(P \cup I\) is equivalent to \(\GroundBottomUp(P,\Head{I}) \cup I\)
  \end{itemize}
\end{frame}
% ----------------------------------------------------------------------
\begin{frame}[fragile,shrink]{Bottom up grounding, step 1}{Hamiltonian cycle}
  \begin{SemiVerbatim}[\small]{.9}
    {\color{comment}\% Step 1}
    path(a,b) :- not omit(a,b), \alert{edge(a,b)}.
    \(\vdots\) {\color{comment}\% 7 rules total}
    path(d,a) :- not omit(d,a), \alert{edge(d,a)}.

    omit(a,b) :- not path(a,b), \alert{edge(a,b)}.
    \(\vdots\) {\color{comment}\% 7 rules total}
    omit(d,a) :- not path(d,a), \alert{edge(d,a)}.

    :- \alert{node(a)}, not on_path(a).  :- \alert{node(b)}, not on_path(b).
    :- \alert{node(c)}, not on_path(c).  :- \alert{node(d)}, not on_path(d).

    :- \alert{node(a)}, not reach(a).  :- \alert{node(b)}, not reach(b).
    :- \alert{node(c)}, not reach(c).  :- \alert{node(d)}, not reach(d).

    reach(a) :- \alert{start(a)}.
  \end{SemiVerbatim}
\end{frame}
% ----------------------------------------------------------------------
\begin{frame}[fragile,shrink]{Bottom up grounding, step 2}{Hamiltonian cycle}
  \begin{SemiVerbatim}[\small]{.9}
    {\color{comment}\% Step 2 \alert{and} rules of Step 1}
    :- \alert{path(a,c)}, \alert{path(b,c)}, a < b.
    :- \alert{path(b,d)}, \alert{path(c,d)}, b < c.
    :- \alert{path(c,a)}, \alert{path(d,a)}, c < d.

    :- \alert{path(a,b)}, \alert{path(a,c)}, b < c.
    :- \alert{path(c,a)}, \alert{path(c,d)}, a < d.
    :- \alert{path(b,c)}, \alert{path(b,d)}, c < d.

    on_path(a) :- \alert{path(c,a)}, \alert{path(a,c)}.
    \(\vdots\) {\color{comment}\% 12 rules total}
    on_path(d) :- \alert{path(c,d)}, \alert{path(d,a)}.

    reach(b) :- \alert{reach(a)}, \alert{path(a,b)}.
    reach(c) :- \alert{reach(a)}, \alert{path(a,c)}.
  \end{SemiVerbatim}
\end{frame}
% ----------------------------------------------------------------------
\begin{frame}[fragile,shrink]{Bottom up grounding, step 3 and 4}{Hamiltonian cycle}
  \bigskip
  \begin{SemiVerbatim}{.9}
    {\color{comment}\% Step 3 \alert{and} rules of Step 2}
    reach(c) :- \alert{reach(b)}, path(b,c).
    reach(d) :- \alert{reach(b)}, path(b,d).
    reach(a) :- \alert{reach(c)}, path(c,a).
    reach(d) :- \alert{reach(c)}, path(c,d).

      {\color{comment}\% Step 4 \alert{and} rules of Step 3}
    reach(a) :- \alert{reach(d)}, path(d,a).
  \end{SemiVerbatim}
\end{frame}
% ----------------------------------------------------------------------
\begin{frame}{Properties of bottom-up grounding}
  \bigskip
  \begin{itemize}
    \item Grounds only \alert{relevant} rules
          \begin{itemize}\normalsize
            \item each positive body literal has a non-cyclic derivation \\
                  (ignoring negative literals)
          \end{itemize}
          \smallskip
    \item \alert{Re-grounds} rules from previous steps
          \medskip
    \item Performs no \alert{simplifications}
  \end{itemize}
\end{frame}
% ----------------------------------------------------------------------
\begin{frame}{Improving bottom-up grounding}
  \bigskip
  \begin{itemize}
    \item Use dependencies to \alert{focus} grounding
          \begin{itemize}\normalsize
            \item begin with partial atom base given by facts
            \item use rule dependency graph of program to obtain \alert{components} that can be \alert{grounded successively}
          \end{itemize}
          \medskip
    \item Adapt \alert{semi-naive evaluation} put forward in the database field
          \begin{itemize}\normalsize
            \item avoids redundancies when grounding
          \end{itemize}
          \medskip
    \item Perform \alert{simplifications} during grounding
          \begin{itemize}\normalsize
            \item remove literals from rule bodies if possible
            \item omit rules if body cannot be satisfied
          \end{itemize}
  \end{itemize}
\end{frame}
% ----------------------------------------------------------------------
\begin{frame}{Program dependencies}
  \smallskip
  \begin{itemize}
    \item<2-> \structure{Dependency graph} of program \(P\)
      \begin{itemize}\normalsize
        \item<only@-3> rule \(r_2\) \alert{depends} on rule \(r_1\)
          \\if \(b\in\pbody{r_2}\cup\nbody{r_2}\) unifies with \(h\in\head{r_1}\)
          \smallskip
        \item \(G_P=(P,E)\) where \(E=\{ (r_1,r_2) \mid r_2 \text{ depends on } r_1 \}\)
      \end{itemize}
      \medskip
    \item<3-> \structure{Positive dependency graph} of program \(P\)
      \begin{itemize}\normalsize
        \item<only@-3> rule \(r_2\) \alert{positively depends} on rule \(r_1\)
          \\if \(b\in\pbody{r_2}\) unifies with \(h\in\Head{r_1}\)
          \smallskip
        \item \(G^+_P=(P,E)\) where \(E=\{ (r_1,r_2) \mid r_2 \text{ positively depends on } r_1 \}\)
      \end{itemize}
      \medskip
    \item<only@5-> \structure{Strongly connected components} of a directed graph form a partition into sub-graphs
      in which each node is reachable from any other node
      \medskip
    \item<only@6-> \structure{Topological ordering} \
      of strongly connected components
      \begin{itemize}\normalsize
        \item<7-> \((C_1,\dots,C_n)\) is a topological ordering of \(G_P\)\only<7>{,}
          \smallskip
          \item[]<only@7>
          that is,
          \(
          (r_1,r_2)\in E,
          r_1\in C_i,
          r_2\in C_j,
          \)
          implies
          \(
          i\leq j\)
          \smallskip
        \item<only@8-> \((C_{i,1},\dots,C_{i,m_i})\) is a topological ordering of each \(G^+_{C_i}\)
      \end{itemize}
      \item<only@9->[]\structure{\itarrow} \
      \(
      L_P = (C_{1,1},\dots,C_{1,m_1},\dots,C_{n,1},\dots,C_{n,m_n})
      \)
  \end{itemize}
\end{frame}
% ----------------------------------------------------------------------
\begin{frame}[shrink]{Dependencies}{Hamiltonian cycle}
  \begin{center}
    \DepGraph
  \end{center}
\end{frame}
% ----------------------------------------------------------------------
\begin{frame}{Grounding with dependencies}
  \bigskip
  \begin{center}
    \EmphAlgo{.8}{
      \Fn{\(\GroundWithDependencies(P,D)\)}{
        \(G \leftarrow \emptyset\)\;
        \ForEach{\alert{\(C\) \In \(L_P\)}}{
          \(G' \leftarrow \GroundBottomUp(C,D)\)\;
          \((G,D) \leftarrow (G \cup G', D \cup \Head{G'})\)\;
        }
        \Return \(G\)\;
      }
    }
  \end{center}
  \bigskip
  \begin{itemize}
    \item<2-> \structure{Property} \ Given a safe normal program \(P\) and set of facts \(I\),

      \(P \cup I\) is equivalent to \(\GroundWithDependencies(P,\Head{I}) \cup I\)
  \end{itemize}
\end{frame}
% ----------------------------------------------------------------------
\begin{frame}[fragile,shrink]{Grounding with dependencies}{Hamiltonian cycle}
  \newcommand{\Component}[1]{\(C_{#1}\)}
  \begin{SemiVerbatim}{.9}
    {\color{comment}\% component \Component{1,1}}
    omit(a,b) :- not path(a,b), \alert{edge(a,b)}.
    \(\vdots\) {\color{comment}\% 7 rules total}
    omit(d,a) :- not path(d,a), \alert{edge(d,a)}.

    {\color{comment}\% component \Component{1,2}}
    path(a,b) :- not omit(a,b), \alert{edge(a,b)}.
    \(\vdots\) {\color{comment}\% 7 rules total}
    path(d,a) :- not omit(d,a), \alert{edge(d,a)}.

    ...
  \end{SemiVerbatim}
  \vspace{-.8cm}
  \begin{itemize}
    \item<2-> No re-grounding if there is no positive recursion in a component
  \end{itemize}
  \smallskip
\end{frame}
% ----------------------------------------------------------------------
\begin{frame}[fragile,shrink]{Grounding component \(C_{7,1}\)}{Hamiltonian cycle}
  \begin{SemiVerbatim}{.9}
    {\color{comment}\% Step 1}
    reach(b) :- \alert{reach(a)}, \alert{path(a,b)}.
    reach(c) :- \alert{reach(a)}, \alert{path(a,c)}.

    {\color{comment}\% Step 2 \alert{and} rules of Step 1}
    reach(c) :- \alert{reach(b)}, path(b,c).
    reach(d) :- \alert{reach(b)}, path(b,d).
    reach(a) :- \alert{reach(c)}, path(c,a).
    reach(d) :- \alert{reach(c)}, path(c,d).

      {\color{comment}\% Step 3 \alert{and} rules of Step 2}
    reach(a) :- \alert{reach(d)}, path(d,a).

      {\color{comment}\% less re-grounding but still...}
  \end{SemiVerbatim}
\end{frame}
% ----------------------------------------------------------------------
%
%%% Local Variables:
%%% mode: latex
%%% TeX-master: "../../main"
%%% End:
